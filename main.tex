\documentclass[a4paper]{ctexart}
\usepackage{syntonly}
%令LaTeX编译后不生成PDF文档,只排查错误,编译速度会快不少
\usepackage{amsmath,amssymb,amsfonts,bm,varwidth,enumitem,cases,graphicx,float,subfigure,siunitx}
%graphicx支持插图
%array对表格列格式的扩展
%booktabs排版三线表
%amsmath,amssymb,amsfonts排版数学公式三件套
%mathtools数学公式扩展宏包,提供了公式编号定制和更多的符号、矩阵等
%cases提供numcases环境,在公式的每行case后面增加编号
%varwidth保持盒子的自然宽度
\usepackage[most]{tcolorbox}
%tcolorbox以TikZ为基础提供排版样式丰富的彩色盒子的功能,并加载大部分的库
\usepackage[mathstyleoff]{breqn}
%breqn自动对长的显示公式折行,且会随行宽而自动调整折行的位置,并启用安全模式
\usepackage[left=19.1mm,right=19.1mm,top=1in,bottom=1in]{geometry}
%geometry修改页边距参数
\ctexset{section/name={第,节}}
%修改章节编号的样式
\usepackage[pdftitle=平面几何讲义,colorlinks=true,linkcolor=black,bookmarksnumbered=false,pdfsubject=平面几何,pdfstartview=FitH]{hyperref}
%colorlinks=true为链接文字带颜色
%bookmarksnumbered=true书签带章节编号
%pdfauthor作者
%pdfsubject主题
%pdfstartview=FitH设置PDF页面以适合宽度的方式显示
\allowdisplaybreaks[4]%允许多行公式分页
\graphicspath{{figures/}}
\title{\bf{平面几何、解析几何讲义}}
\author{张杰骜\\2250368776@qq.com}
\date{\today}

\begin{document}
\maketitle%生成标题页
\begin{center}%---------------------------------------------------------
    \includegraphics[scale=0.2]{0}
\end{center}%---------------------------------------------------------
\newpage
\tableofcontents%生成目录
\newpage
\raggedbottom%令页面在垂直方向向顶部对齐

%\section{圆的初步}
\subsection{四点共圆的判定}
四点共圆的条件:四点A,B,C,D共圆的充要条件为:$\angle{ABC}=\angle{ADC}\neq0$(同侧)
或$\angle{ABC}+\angle{ADC}=180^{\circ}$(异侧).

此外,相交弦定理逆定理,切割线定理逆定理,托勒密定理逆定理等许多关于圆的相关定理的逆定理
也都可作为四点共圆的判据,需要根据题目条件灵活使用(不过使用做多的还是利用角度进行判断)。

四点共圆在平几题中出现频率非常高,在问题中一般有两种形式:一是作为证题的目的;通常需要根据
题设条件推出判定四点共圆的条件(前述的几种),也可以用两组四点共圆来证明题设要求的四点共圆;
二是作为解题的工具,可以用于证明角相等、线垂直等。
\begin{figure}[H]
    \centering  
    \subfigure{
    \includegraphics[scale=0.4]{4}}
    \subfigure{
    \includegraphics[scale=0.4]{5}}
\end{figure}
\subsection{与圆相关的定理}
(1)托勒密定理:在凸四边形ABCD中,$AB\cdot CD+AD\cdot BC\geq AC\cdot BD$,
当且仅当四边形ABCD是圆内接四边形时,等号成立。
\begin{center}
    \includegraphics*[scale=0.5]{1.png}
\end{center}

三弦定理:设PA,PB,PC是圆内一公共端点的三条弦,则
$$PA\sin{\angle{BPC}}+PC\sin{\angle{APB}}=PB\sin{\angle{APC}}$$

三弦定理逆定理:设PA,PB,PC是一公共端点的三条线段,若
$$PA\sin{\angle{BPC}}+PC\sin{\angle{APB}}=PB\sin{\angle{APC}}$$
则P,A,B,C四点共圆。
\begin{center}
    \includegraphics*[scale=0.3]{2.png}
\end{center}

(2)西姆松定理
过三角形外接圆上异于三角形顶点的任意一点作三边的垂线,则三垂足点共线(此线称为西姆松线)。
\begin{center}
    \includegraphics*[scale=0.5]{3.png}
\end{center}

(3)帕斯卡定理
圆内接六边形ABCDEF三组对边(延长线)的交点P、Q、R三点共线。
\begin{figure}[H]
    \centering  
    \subfigure{
    \includegraphics[scale=0.5]{6}}
    \subfigure{
    \includegraphics[scale=0.5]{7}}
\end{figure}
\newpage
\subsection{例题}
1.$\quad$ AB是圆O的直径.C与D是互异的圆O上的两点,且在AB的一侧.过C、D作圆的切线交于点E.线段AD与BC交于点F,直线EF交AB于M.求证:E,C,M,D四点共圆.
\begin{center}
    \includegraphics*[scale=0.4]{20}
\end{center}

~\\
~\\
~\\
~\\
~\\
~\\

2.A、B、C三点共线,O点在直线外。$O_1,O_2,O_3$分别
为$\bigtriangleup OAB,\bigtriangleup OBC,\bigtriangleup OCA$的外心.求证: 
$O,O_1,O_2,O_3$四点共圆.
\begin{center}
    \includegraphics*[scale=0.4]{21}
\end{center}
\newpage

3.$\bigtriangleup ABC$中,$AB=AC$,点E,F分别在AB,AC上,$AE<AF$,BF,CE交于点P.求证:$PF<PE$.
\begin{center}
    \includegraphics*[scale=0.4]{22}
\end{center}

~\\

4.圆O过$\bigtriangleup ABC$顶点A、C,且与AB、BC交于K、N(K、N不同).$\bigtriangleup ABC$外接圆和$\bigtriangleup BKN$外接圆交于B、M两点。求证:
$\angle{BMO}=90^{\circ}$.
\begin{center}
    \includegraphics*[scale=0.4]{23}
\end{center}


5.在平行四边形ABCD中,过点C作AB,AD的垂线CM,CN,垂足分别为M,N,MN,BD的延长线交于点P.求证:PC$\bot $AC.
\begin{center}
    \includegraphics*[scale=0.4]{24}
\end{center}


%\newpage
\section{三角形的五心}
\subsection{五心的定义与性质}
三角形的重心、垂心、内心、外心、旁心称为三角形的五心。


1.重心:三角形的三条中线交于一点,该点叫做三角形的重心;重心将每条中线都分成定比2:1。

$\bigtriangleup{ABC}$中线长度公式:
$$AD^2=\frac{1}{2}\sqrt{2AC^2+2AB^2-BC^2}$$,AD为中线。


2.外心:三角形外接圆的圆心,叫做三角形的外心。



3.垂心:三角形的三条高交于一点,该点叫做三角形的垂心。
\begin{center}
    \includegraphics*[scale=0.75]{8}
\end{center}

垂心的性质:

(1)三角形的三个顶点、三个垂足、垂心这7点可以得到6个四点圆,
三组相似三角形,且$AH\cdot HD=BH\cdot HE=CH\cdot HF$。

(2)H、A、B、C四点中任一点是其余三点为顶点的三角形的垂心,称为一垂心组。

(3)$\angle{BHC}=180^{\circ}-\angle{A}=\angle{B}+\angle{C}$

(4)O是外心,H是垂心,则$\angle{BAO}=\angle{HAC},\angle{ABO}=\angle{HBC}$

(5)H关于三边的对称点在$\bigtriangleup{ABC}$的外接圆上,关于
三边中点的对称点在$\bigtriangleup{ABC}$的外接圆上。

(6)三角形外心O、重心G、垂心H三点共线,且OG:GH=1:2(欧拉线)。

(7)三角形任一顶点到垂心的距离等于外心到对边距离的2倍。

(8)设$\bigtriangleup{ABC}$的垂心为H,外接圆半径为R,
则
$$
\frac{HA}{|\cos{A}|}=\frac{HB}{|\cos{B}|}=\frac{HC}{|\cos{C}|}=2R
$$

(9)三角形的垂心是其垂足三角形的内心。

九点圆:$\bigtriangleup{ABC}$三条高的垂足$D,E,F$,
三边中点$A_1,B_1,C_1$,以及顶点与
垂心H联线段的中点$A_2,B_2,C_2$九点共圆。
\begin{center}
    \includegraphics[scale=0.5]{9.png}
\end{center}

证明:
\begin{figure}[H]
    \centering  
    \subfigure{
        \begin{minipage}[t]{0.25\linewidth}
            \centering
            \includegraphics[scale=0.35]{10}
        \end{minipage}}
    \subfigure{
        \begin{minipage}[t]{0.25\linewidth}
            \centering
            \includegraphics[scale=0.35]{11}
        \end{minipage}}
    \subfigure{
        \begin{minipage}[t]{0.25\linewidth}
            \centering
            \includegraphics[scale=0.35]{12}
        \end{minipage}}
\end{figure}

位似性质:
\begin{center}
    \includegraphics*[scale=0.75]{13}
\end{center}
\newpage
4.内心:三角形内切圆的圆心,叫做三角形的内心(三角平分线的交点)。

性质:

(1)张角公式:$\angle{BIC}=90^{\circ}+\frac{1}{2}\angle{A}$


(2)设I为$\bigtriangleup ABC$的内心,$AI,BI,CI$分别交外接圆于$D,E,F$,则I为$\bigtriangleup{DEF}$的垂心。
\begin{center}
    \includegraphics*[scale=0.5]{15}
\end{center}

(3)设I为$\bigtriangleup ABC$的内心,$BC=a,AC=b,AB=c$,
$\angle{A}$的平分线交BC于K,交$\bigtriangleup ABC$的外接圆于点D,则
\[
    \frac{AI}{KI}=\frac{DI}{DK}=\frac{AD}{DI}=\frac{b+c}{a}
\]
\begin{center}
    \includegraphics*[scale=0.6]{16}
\end{center}

5.旁心:三角形的旁切圆的圆心,叫做三角形的旁心(三角形一内角平分线和另两顶点处的外角平分线的交点)

(1)鸡爪定理:设$I,I_1$分别是$\bigtriangleup{ABC}$的内心、旁心,延长AI交外接圆于S点,则
$$SI=SB=SC=SI_1$$
\begin{center}
    \includegraphics*[scale=0.5]{17}
\end{center}

(2)三角形的三个旁心与内心构成一垂心组。
\begin{figure}[H]
        \centering  
        \subfigure{
                \includegraphics[scale=0.5]{18}}
        \subfigure{
                \includegraphics[scale=0.5]{19}}
\end{figure}

\subsection{例题}
1.设H为锐角三角形$\bigtriangleup ABC$的垂心,D为边BC的中点,过点H的直线分别交边AB、AC于点F、E,使得AE=AF,射线DH与$\bigtriangleup ABC$的外接圆交于P.求证:
P、A、E、F四点共圆.

\begin{center}
    \includegraphics*[scale=0.3]{25}
\end{center}

2.设点O是锐角三角形$\bigtriangleup ABC$的外心.分别以$\bigtriangleup ABC$三边的中点为圆心作过点O的圆,这三个圆两两的异于O的交点分别
为K、L、M.证明:点O是$\bigtriangleup KLM$的内心.
\begin{center}
    \includegraphics*[scale=0.3]{26}
\end{center}
\newpage
3.在$\bigtriangleup ABC$中,AB=AC,一个圆内切于$\bigtriangleup ABC$的外接圆$\odot O$于M,并与AB、AC分别相切于P、Q两点.I为线段PQ中点.求证:
I是$\bigtriangleup ABC$的内心.
\begin{center}
    \includegraphics*[scale=0.3]{27}
\end{center}

4.$AD$是直角三角形$ABC$斜边上的高,($AB<AC$),$I_1,I_2$分别是$\bigtriangleup ABD,\bigtriangleup ACD$的内心,$\bigtriangleup AI_1I_2$的外接圆$\odot O$分别交$AB,AC$
于$E,F$,直线$EF,BC$交于点$M$.证明:$I_1,I_2$分别是$\bigtriangleup ODM$的内心与旁心.
\begin{center}
    \includegraphics*[scale=0.4]{28}
\end{center}

5.$\bigtriangleup ABC$的内切圆$I$切$BC,AC$于$D,E$,$K,L$为$AB,AC$中点,证明:$BI,KL,DE$三线共点
\begin{center}
    \includegraphics*[scale=0.4]{29}
\end{center}
\newpage

6.在锐角三角形$ABC$中,$AB<AC$,$O$为外心,$H$为垂心。设直线$AO$与$BC$交于点$D$,点$E$在边BC上且满足$HE // AD$.证明:
$BE=CD$.
\begin{center}
    \includegraphics*[scale=0.4]{30}
\end{center}

7.例题3的推广:

(1)一般情形:设$\odot O'$与$\bigtriangleup ABC$外接圆切于点$D$,与$AB,AC$切于$E,F$点,$I$为线段$EF$中点.证明:$I$为
$\bigtriangleup ABC$的内心.
\begin{center}
    \includegraphics*[scale=0.4]{31}
\end{center}

(2)旁心情形:设$\odot O'$与$\bigtriangleup ABC$外接圆切于点$D$,与$AB,AC$的延长线切于$E,F$点,$I'$为线段$EF$中点.证明:$I'$为
$\bigtriangleup ABC$的$BC$边外的旁心.
\begin{center}
    \includegraphics*[scale=0.4]{32}
\end{center}

8.设$\bigtriangleup ABC$的垂心、外心分别为点$H,O$,BC的中垂线交$\bigtriangleup AOH$的外接圆于$A_1$,
($A_1\neq O$).类似定义点$B_1,C_1$。求证:$AA_1,BB_1,CC_1$三线共点.
\newpage
\section{圆中的比例线段、根轴}
\subsection{圆幂}
1.相交弦定理:圆内的两条相交弦被交点分成的两条线段的积相等。

2.切割线定理:从圆外一点引圆的切线和割线,切线长是这点到割线与圆交点的两条线段长
的比例中项

3.割线定理:从圆外一点引圆的两条割线,这一点到每条割线与圆的交点的两条线段长的积
相等

上述三个定理统称为圆幂定理:过一定点作两条直线与圆相交,则定点到每条直线与圆的交点的两条线段的
积相等,即它们的积为定值。
\begin{center}
    \includegraphics*[scale=0.4]{33}
\end{center}

4.点到圆的幂:点到圆心的距离为$d$,圆的半径为$r$,称$d^2-r^2$为定点到圆的幂。

当定点在圆内时,$d^2-r^2<0$

当定点在圆上时,$d^2-r^2=0$

当定点在圆外时,$d^2-r^2>0$

\subsection{根轴}
1.根轴:到两圆等幂的点的轨迹是与此二圆圆心连心线垂直的一条直线,这条直线称为两圆的根轴。

2.如果两圆相交,其根轴为两圆公共弦所在的直线。

3.如果两圆相切,其根轴就是过两圆切点的公切线。

4.三个圆,其两两的根轴或交于一点(根心),或互相平行。当三个圆两两相交时,三条公共弦所在的直线交于一点,
这一性质可以用于判断三线共点。
\begin{center}
    \includegraphics*[scale=0.3]{34}
\end{center}

\subsection{例题}
1.ABCD是圆$O$的内接四边形,延长$AB,DC$交于点$E$,
$EP,FQ$分别切圆$O$于$P,Q$.证明:$EP^2+FQ^2=EF^2$.

\begin{center}
    \includegraphics*[scale=0.3]{35}
\end{center}

2.自圆外一点P向圆O作切线PA、PB,A、B为切点,AB与PO
相交于C,弦EF过点C.证明:$\angle{APE}=\angle{BPF}$
\begin{center}
    \includegraphics*[scale=0.4]{36}
\end{center}

3.自圆外一点P向圆O引割线交圆于R、S两点,又作切线PA、PB,
A、B为切点,AB与PR相交于Q.证明:$\frac{1}{PR}+\frac{1}{PS}=\frac{2}{PQ}$
\begin{center}
    \includegraphics*[scale=0.35]{37}
\end{center}
\newpage
4.圆内接四边形ABCD的对角线交于点K,点M和N分别是
对角线AC和BD的中点,$\bigtriangleup ADM,\bigtriangleup BCM$的外接圆交于
点M、L,证明:K、L、M、N四点共圆.

~\\
~\\
~\\
~\\
~\\
~\\
~\\
~\\
~\\
~\\
~\\
~\\

5.$\bigtriangleup ABC$的内切圆与AB切于点C'
,设$\bigtriangleup ACC'$的内切圆分别与AB、AC切于点
$C_1,B_1$,$\bigtriangleup BCC'$的内切圆分别与AB、BC切于点$C_2,A_2$.
证明:$B_1C_1,A_2C_2,CC'$三线共点.
~\\
~\\
~\\
~\\
~\\
~\\
~\\
~\\
~\\
~\\
~\\

6.$\bigtriangleup ABC$中,E、F分别为AB、AC中点,CM、BN为高,
EF交MN于P,O、H分别为三角形的外心与垂心.证明:$AP\bot OH$.
\newpage

7.(欧拉定理)O、I分别为$\bigtriangleup ABC$的外心、内心,R、r分别为$\odot O,\odot I$的半径.
证明:$OI^2=R^2-2Rr$.
\begin{center}
    \includegraphics*[scale=0.5]{38}
\end{center}

8.已知D是$\bigtriangleup ABC$外接圆上任一点,作DE、DF与内切圆I都相切,交外接圆于E、F.
求证:EF也与内切圆I相切.
\begin{center}
    \includegraphics*[scale=0.5]{39}
\end{center}

9.$A_1A_2$是两个相离的圆$\odot O_1,\odot O_2$的外公切线,设$A_1A_2$的中点为$K$,过$K$作
$\odot O_1 ,\odot O_2$的切线$KB_1,KB_2$,$B_1,B_2$为切点.
直线$A_1B_1,A_2B_2$交于点$L$,直线$O_1O_2,KL$交于点$P$.
证明:$B_1,B_2,P,L$四点共圆.
\begin{center}
    \includegraphics*[scale=0.5]{40}
\end{center}
\newpage
\section{简单实战训练}
1.在$\bigtriangleup ABC$中,M是边AC的中点,D、E是$\bigtriangleup ABC$的外接圆在点A处的切线上的两点,
满足$MD//AB$,且A是线段DE的中点,过A、B、E三点的圆与边AC相交于另一点P,过
A、D、P三点的圆与DM延长线相交于点Q.证明:$\angle{BCQ}=\angle{BAC}$.
\begin{center}
    \includegraphics*[scale=0.35]{41}
\end{center}

2.如图所示,I是$\bigtriangleup ABC$的内心,点P、Q分别为I在边AB、AC上的投影,
直线PQ与$\bigtriangleup ABC$的外接圆相交于X、Y(P在X、Q之间)。已知
B、I、P、X四点共圆,证明:C、I、Q、Y四点共圆.
\begin{center}
    \includegraphics*[scale=0.35]{42}
\end{center}

3.如图,在$\bigtriangleup ABC$,$AB>AC$,$\bigtriangleup ABC$内两点X,Y均在$\angle{BAC}$的
平分线上,且满足$\angle{ABX}=\angle{ACY}$,设BX的延长线与线段CY交于点P,
$\bigtriangleup BPY$的外接圆与$\bigtriangleup CPY$的外接圆交于P与另一点
Q.证明:A、P、Q三点共线.
\begin{center}
    \includegraphics*[scale=0.35]{43}
\end{center}

4.如图所示,在锐角$\bigtriangleup ABC$中,$AB>AC$,M是$\bigtriangleup ABC$的外
接圆的劣弧$\hat{BC}$的中点,K是$\bigtriangleup BAC$的外角平分线与BC延长线的交点,在过点A且
垂直于BC的直线上取一点D(异于A),使得DM=AM.设$\bigtriangleup ADK$的外接圆与$\bigtriangleup ABC$外接圆相
交于点A及另一点T.证明:AT平分线段BC.
\begin{center}
    \includegraphics*[scale=0.35]{44}
\end{center}

5.如图,在等腰$\bigtriangleup ABC$中,$AB=BC$,I为内心,M为BI的中点,
P为边AC上一点,满足$AP=3PC$,PI延长线一点H满足$MH\perp PH$,Q为
$\bigtriangleup ABC$的外接圆上劣弧AB的中点.证明:$BH\perp QH$.
\begin{center}
    \includegraphics*[scale=0.30]{45}
\end{center}

6.如图,$\bigtriangleup ABC$为锐角三角形,AB<AC,M为BC边
的中点,点D和E分别为$\bigtriangleup ABC$的外接圆$\hat{BAC}$和$\hat{BC}$的中点,F为$\bigtriangleup ABC$的内
切圆在AB边上的切点,G为AE与BC的交点,N在线段EF上,满足$NB\perp AB$.
证明:若$BN=EM$,则$DF\perp FG$.
\begin{center}
    \includegraphics*[scale=1.0]{46}
\end{center}
\newpage
\section{平面几何中的常见方法——代数法、三角法}

\subsection{代数法常用定理与结论}

比例线段、切线长公式(内心,旁心)、角平分线定理、梅涅劳斯定理、塞瓦定理、托勒密定理、
圆幂定理、调和性质等等。(还有一些几何结构中表现出的与线段长、线段比例相关的性质)

\subsection{代数法例题}
1.已知O、I分别为$\bigtriangleup ABC$的外心和内心,$BC=a,CA=b,AB=c$.问当且仅当$a,b,c$满足什么条件时,有$OI\perp BI$.
\begin{flushleft}
    \includegraphics*[scale=0.5]{51}
\end{flushleft}
~\\
~\\
~\\


2.AD是$\bigtriangleup ABC\angle BAC$的外角平分线,并交$\bigtriangleup ABC$的外界圆
于D,以CD为直径的圆分别交BC,AC于P,Q.证明:线段PQ把$\bigtriangleup ABC$的周长二等分.
\begin{flushleft}
    \includegraphics*[scale=0.5]{47}
\end{flushleft}
~\\
~\\
\newpage

3.$\bigtriangleup ABC$中,BE、CF为AC、AB边上的高,$FP\perp BC$于P,$FQ\perp AC$于Q,
$EM\perp BC$于M,$EN\perp AB$于N,且$FP+FQ=CF,\quad EM+EN=BE$.证明:$\bigtriangleup ABC$
为正三角形.
\begin{flushleft}
    \includegraphics*[scale=0.5]{48}
\end{flushleft}
~\\
~\\



4.设AB是$\odot O$的直径,过点A、B的切线分别为$l_a,l_b$,C是
圆周上任意一点,BC交$l_a$于K,$\angle CAK$的平分线交CK于H.设M是弧CAB的中点,
HM与$\odot O$交于点S,过点M的切线与$l_b$交于点T.证明:S、T、K三点共线.
\begin{flushleft}
    \includegraphics*[scale=0.5]{49}
\end{flushleft}
~\\
~\\
~\\

\subsection{三角法}
1.常用方法与公式:正弦定理、余弦定理、积化和差、和差化积、三倍角公式、三弦定理、张角定理等.

2.三角形中的恒等式:
\begin{equation*}
    \cos^2 A+\cos^2 B+\cos^2 C=1-2\cos A \cos B \cos C 
\end{equation*}

\begin{equation*}
    \tan A+\tan B+\tan C=\tan A \tan B \tan C
\end{equation*}

\begin{equation*}
    \tan{\frac{A}{2}} \tan{\frac{B}{2}}+\tan{\frac{B}{2}} \tan{\frac{C}{2}}+\tan{\frac{A}{2}} \tan{\frac{C}{2}}=1
\end{equation*}

3.设$r,R$分别为$\bigtriangleup ABC$的内切圆、外接圆半径,则有
\begin{equation*}
    \frac{r}{R}=4\sin{\frac{A}{2}} \sin{\frac{B}{2}} \sin{\frac{C}{2}}=\cos A + \cos B +\cos C-1
\end{equation*}

4.张角定理:若A、B、C分别是线束$Px,Py,Pz$上三点,则A、B、C三点共线的充要条件是:
$$\frac{\sin{\beta}}{PA} + \frac{\sin{\alpha}}{PB}=\frac{\sin{(\alpha+\beta)}}{PC}$$
\begin{center}
    \includegraphics*[scale=0.6]{50}
\end{center}

5.证明两角$\alpha,\beta$相等的方法:
\[
    \frac{\sin{(\alpha+\theta)}}{\sin{\alpha}}=\frac{\sin{(\beta+\theta)}}{\sin{\beta}}\quad \quad \quad (\theta \neq 0)
\]
\subsection{三角法例题}
1.求$\bigtriangleup ABC$旁切圆半径$r_a,r_b,r_c$与外接圆半径$R$的关系(用三角函数表示)
\newpage
2.O、I分别为O、I分别为$\bigtriangleup ABC$的外心和内心,AD是BC边上的高,O、I、D三点共线.证明:$\bigtriangleup ABC$外接圆半径$R$等于
BC边上旁切圆的半径$r_a$.
\begin{flushleft}
    \includegraphics*[scale=0.5]{52}
\end{flushleft}
~\\
~\\
~\\
~\\
~\\

3.AB为圆O的直径,直线$l$切圆O于A.
C、M、D在直线$l$上满足$CM=DM$,
又设BC、BD交圆O于P、Q,圆O切线PR、QR交于点R.
证明:B、M、R三点共线.
\begin{flushleft}
    \includegraphics*[scale=0.4]{53}
\end{flushleft}
\newpage

4.圆P与圆Q分别于圆O内切于点A、B,且圆P与圆Q外切.记圆P与圆Q的内公切线
与直线AB相交于点C.证明:OC平分线段PQ.
\begin{flushleft}
    \includegraphics*[scale=0.35]{54}
\end{flushleft}

5.已知$\bigtriangleup ABC$的一旁切圆与CA,CB延长线相切于点P、Q,另一旁切圆
与BA,BC延长线相切于点S,T.延长QP,TS交于点M.证明:$MA\perp BC$.
\begin{flushleft}
    \includegraphics*[scale=0.35]{55}
\end{flushleft}
~\\

6.已知$\odot I$的外切四边形ABCD,I在AC上的垂足为K.证明:$\angle BKC=\angle DKC$.
\begin{flushleft}
    \includegraphics*[scale=0.35]{57}
\end{flushleft}


\newpage
\section{线段、角相等与图形相似、全等问题}
线段相等的证明方法:全等、等弧对等弦、比例线段法、代数法、张角定理、面积法等,
需要根据题目条件灵活运用。

角度相等的证明方法:全等或相似、四点共圆、角平分线与内心性质、三角法、鸡爪定理逆定理等。


1.在四边形ABCD中,对角线AC平分$\angle BAD$,在CD上取一点E,BE与AC
相交于F,延长DF交BC于G.证明:$\angle GAC=\angle EAC$.
~\\
~\\
~\\
~\\
~\\
~\\
~\\
~\\
~\\
~\\
~\\
~\\
~\\
~\\
~\\

2.设D为$\bigtriangleup ABC$的边AC上一点,E、F分别为线段BD和AC上的点,满足$\angle BAE=\angle CAF$.
再设P,Q为线段BC和BD上的点,使得$EP//QF//DC$.证明:$\angle BAP=\angle QAC$

\newpage

3.$\odot O_1,\odot O_2$相交于点M、N,设$l$为两圆的两条公切线中距离
点M较近的那条.$l$与两圆分别切于A、B两点.设经过点M且与$l$平行的直线与
两圆分别交于C、D(不同于M)两点.直线CA与DB交于点E,直线AN和CD相交于点
P,直线BN和CD相交于点Q.证明:EP=EQ.
\begin{flushleft}
    \includegraphics*[scale=0.35]{58}
\end{flushleft}
~\\

4.四边形ABCD内接于圆,P是AB的中点,$PE\perp AD,PF\perp BC,PG\perp CD$,
M是线段PG和EF的交点.证明:$ME=MF$.
\begin{flushleft}
    \includegraphics*[scale=0.35]{59}
\end{flushleft}

5.(牛顿线)完全四边形ABCD,三对角线的中点分别为M、N、T.证明:M、N、T三点共线.
\begin{flushleft}
    \includegraphics*[scale=0.35]{60}
\end{flushleft}


\newpage
\section{直线平行或垂直问题}
\subsection{平行问题}
1.菱形ABCD的内切圆O与各边分别切于E、F、G、H,在弧EF与弧GH上
分别作圆O切线交AB于M,交BC于N,交CD于P,交DA于Q.
证明:$MQ//NP$.
\begin{flushleft}
    \includegraphics*[scale=0.35]{61}
\end{flushleft}


2.AD和CF是非锐角$\bigtriangleup ABC$的高,H和O分别是$\bigtriangleup ABC$
的垂心和外心,$M$是边AC的中点,直线BO交边AC于P,
直线BH和CF交于Q.证明:直线HM和PQ平行.
\begin{flushleft}
    \includegraphics*[scale=0.25]{62}
\end{flushleft}

3.在$\bigtriangleup ABC$中,AT为$\angle A$的平分线,D,E分别在AB、AC上,
且BD=CE.又BC、DE的中点分别为M和N.证明:$MN//AT$.
\begin{flushleft}
    \includegraphics*[scale=0.3]{63}
\end{flushleft}

\newpage
\subsection{垂直问题}
1.从等腰三角形$ABC$的底边$AC$的中点$M$作边$BC$的垂线$MH$,点$P$是$MH$的中点.证明:
$AH\perp BP$.
\begin{flushleft}
    \includegraphics*[scale=0.3]{68}    
\end{flushleft}

2.圆$O$的弦$AB$和$CD$交于$K$,过各弦的两端作圆的切线分别交于$P,Q$.求证:
$OK\perp PQ$.
\begin{flushleft}
    \includegraphics*[scale=0.35]{69}    
\end{flushleft}
~\\

3.$\bigtriangleup PAB$与$\bigtriangleup QAC$为$\bigtriangleup ABC$外的两个三角形,
满足$AP=AB,AQ=AC,\angle BAP=\angle CAQ$,线段BQ与CP相交于点R,设O是$\bigtriangleup BCR$的
外心.证明:$AO\perp PQ$.
\begin{flushleft}
    \includegraphics*[scale=0.35]{70}    
\end{flushleft}



\newpage
\section{点共线或线共点问题}
\subsection{点共线问题}

1.已知$O$是锐角三角形$ABC$的外心,$BE,CF$为$AC,AB$边上的高,自垂足$E,F$分别作
$AB,AC$的垂线,垂足为$G,H$.$EG,FH$相交于$K$.证明:$A,K,O$三点共线.
\begin{flushleft}
    \includegraphics*[scale=0.35]{72}
\end{flushleft}

2.四边形$BCEF$内接于圆$O$,其边$CE$与$BF$的延长线交于点$A$,由点$A$作圆$O$的两条切线
$AP,AQ$,切点分别为$P,Q$,$BE$与$CF$的交点为$H$.证明:$P,H,Q$三点共线.
\begin{flushleft}
    \includegraphics*[scale=0.35]{71}
\end{flushleft}

3.在锐角$\bigtriangleup ABC$的边$AB,AC$上分别取点$M,N$,分别以$BN,CM$为直径各作一圆,
两圆交于点$P,Q$.证明:$P,Q$以及$\bigtriangleup ABC$的垂心$H$共线.
\begin{flushleft}
    \includegraphics*[scale=0.35]{73}
\end{flushleft}

\newpage

\subsection{线共点问题}
4.设$P$是$\bigtriangleup ABC$内一点,满足$\angle APB-\angle ACB=\angle APC-\angle ABC$.
又设$D,E$分别是$\bigtriangleup APB,\bigtriangleup APC$的内心.证明:$AP,BD,CE$三线共点.
\begin{flushleft}
    \includegraphics*[scale=0.35]{74}
\end{flushleft}
~\\

5.四边形$ABCD$内接于圆$O$,对角线$AC$交$BD$于$P$.设$\bigtriangleup ABP,\bigtriangleup BCP,\bigtriangleup CDP,\bigtriangleup DAP$的外接圆
圆心分别为$O_1,O_2,O_3,O_4$.证明:$OP,O_1O_3,O_2O_4$三线共点.
\begin{flushleft}
    \includegraphics*[scale=0.35]{75}
\end{flushleft}
~\\

6.过$\bigtriangleup ABC$的两顶点$B,C$的圆分别与$AB,AC$相交于$C',B'$.设
$H,H'$分别为$\bigtriangleup ABC,\bigtriangleup AB'C'$的垂心.证明:
$BB',CC',HH'$三线共点.
\begin{flushleft}
    \includegraphics*[scale=0.25]{76}
\end{flushleft}

$(\mathbf{2 0 2 2} \cdot \mathbf{I M O} \cdot \mathbf{D 2} \cdot \mathbf{P 4})$ 
设凸五边形 $A B C D E$ 满足 $B C=D E$. 若在 $A B C D E$ 内存在一点 $T$ 使得 $T B=T D,T C=T E$ 且 $\angle A B T=\angle T E A$. 
直 线 $A B$ 分别与直线 $C D$ 和 $C T$ 交于点 $P$ 和 $Q$ ,且 $P, B, A, Q$ 在同一直线上按此顺序 排列; 
直线 $A E$ 分别与直线 $C D$ 和 $D T$ 交于点 $R$ 和 $S$ ,且 $R, E, A, S$ 在同一直线上 按此顺序排列. 证明: $P, S, Q, R$ 四点共圆.
\begin{flushleft}
    \includegraphics*[scale=0.5]{77}
\end{flushleft}
\newpage
\section*{一试}
\subsection*{一.填空题(每题8分,共64分)}
1.若点$P(x,y)$在直线$x+3y-3=0$上移动,则函数$f(x,y)=3^x+9^y$的最小值为$\underline{5\sqrt[5]{\frac{27}{4}}}$.
~\\

2.由正$4n+2$边形的$4n+2$个顶点构成的梯形的个数为$\underline{2n(2n-1)(2n+1)}$.
~\\

3.设$O$是锐角$\bigtriangleup ABC$的外心,$AB=6,AC=10$.若$\overrightarrow{AO}=x\overrightarrow{AB}+y\overrightarrow{AC}$,且$2x+10y=5$,则$\cos{\angle BAC}=\underline{\frac{1}{3}}$.
~\\

4.数列$\{a_n \}$满足$a_1=\frac{1}{4},a_2=\frac{1}{5}$,且$a_1a_2+a_2a_3+\cdots+a_na_{n+1}=na_1a_{n+1}$对任何正整数$n$成立,则
$\frac{1}{a_1}+\frac{1}{a_2}+\cdots+\frac{1}{a_{97}}$的值为\underline{5044}.
~\\

5.已知正方体$ABCD-A_1B_1C_1D_1$的棱长为1,$O$为底面$ABCD$的中心,M,N分别是棱$A_1D_1,CC_1$的中点,则四面体$OMNB_1$的体积为$\underline{\frac{7}{48}}$.
~\\

6.关于$x$的方程$x^4-2ax^2-x+a^2-a=0$恰有两个不相等的实根,则实数$a$的取值范围是$\underline{(-\frac{1}{4},\frac{3}{4})}$.
~\\

7.已知点$F_1,F_2$分别是中心在坐标原点,对称轴为坐标轴的双曲线$C$的左、右焦点,过点$F_2$的直线$l$交双曲线的右支于$A,B$两点,点$I_1,I_2$分别是
$\bigtriangleup AF_1F_2,\bigtriangleup BF_1F_2$的内心.若双曲线$C$的离心率为2,$|I_1I_2|=\frac{9}{2}$,直线$l$的倾斜角的正弦值为$\frac{8}{9}$,
则双曲线$C$的方程为$\underline{\frac{x^2}{4}-\frac{y^2}{12}=1}$.
~\\

8.用$a_n$表示区间$\left[ 0,1 \right)$内不含数字9的$n$位小数的个数,$S_n$表示这些小数的和.则
$$\lim_{n\to \infty}\frac{S_n}{a_n}=\underline{\frac{4}{9}}.$$


9.(16分)已知$f(x)=\frac{x^4+kx^2+1}{x^4+x^2+1}(k\in R,x\in R)$.

(1)求$f(x)$的最大值和最小值;

(2)求所有的实数$k$,使得对任意的三个实数$a,b,c$,存在一个三角形,其三边长为$f(a),f(b),f(c)$.
\newpage
10.(20分)数列$\{a_n\},n\geq 1$定义为$a_1=1,a_2=4,a_n=\sqrt{a_{n-1}a_{n+1}+1}$.

(1)求证:数列$\{a_n \}$为整数列;

(2)求证:$2a_na_{n+1}+1,n\geq 1$是完全平方数.
\vspace{100mm}

11.(20分)在平面直角坐标系中,点$A,B,C$在双曲线$xy=1$上,满足$\bigtriangleup ABC$为等腰直角三角形.求$\bigtriangleup ABC$面积的最小值.

\newpage
\section*{二试}
%1.(40分)在$\bigtriangleup ABC$中,$\angle BCA$的角平分线与$\bigtriangleup ABC$的外接圆交于点R,与边
%BC的垂直平分线交于点P,与边AC的垂直平分线交于点Q,设点K、L分别是边BC、AC的中点.证明:
%$\bigtriangleup RPK$和$\bigtriangleup RQL$的面积相等.
%\begin{flushleft}
%\includegraphics*[scale=0.35]{66}
%\end{flushleft}
%\vspace{20mm}
1.设$\bigtriangleup ABC$的内心、关于$\angle A$的旁心分别为点$I,I_a$,AE是$\angle ABC$的外接圆的直径,$AD\perp BC$于点$D$.证明:$\angle IEI_a=\angle IDI_a$.
\begin{flushleft}
    \includegraphics*[scale=0.50]{67}
\end{flushleft}
\vspace{20mm}

2.(40分)已知$x_1,x_2,\cdots,x_n$是正实数,求证:
\[
    \frac{1}{1+x_1}+\frac{1}{1+x_1+x_2}+\cdots+\frac{1}{1+x_1+x_2+\cdots+x_n}<\sqrt{\frac{1}{x_1}+\frac{1}{x_2}+\cdots+\frac{1}{x_n}}
    \]



\end{document}