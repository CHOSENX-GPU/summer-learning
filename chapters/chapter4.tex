\section{简单实战训练}
1.在$\bigtriangleup ABC$中,M是边AC的中点,D、E是$\bigtriangleup ABC$的外接圆在点A处的切线上的两点,
满足$MD//AB$,且A是线段DE的中点,过A、B、E三点的圆与边AC相交于另一点P,过
A、D、P三点的圆与DM延长线相交于点Q.证明:$\angle{BCQ}=\angle{BAC}$.
\begin{center}
    \includegraphics*[scale=0.35]{41}
\end{center}

2.如图所示,I是$\bigtriangleup ABC$的内心,点P、Q分别为I在边AB、AC上的投影,
直线PQ与$\bigtriangleup ABC$的外接圆相交于X、Y(P在X、Q之间)。已知
B、I、P、X四点共圆,证明:C、I、Q、Y四点共圆.
\begin{center}
    \includegraphics*[scale=0.35]{42}
\end{center}

3.如图,在$\bigtriangleup ABC$,$AB>AC$,$\bigtriangleup ABC$内两点X,Y均在$\angle{BAC}$的
平分线上,且满足$\angle{ABX}=\angle{ACY}$,设BX的延长线与线段CY交于点P,
$\bigtriangleup BPY$的外接圆与$\bigtriangleup CPY$的外接圆交于P与另一点
Q.证明:A、P、Q三点共线.
\begin{center}
    \includegraphics*[scale=0.35]{43}
\end{center}

4.如图所示,在锐角$\bigtriangleup ABC$中,$AB>AC$,M是$\bigtriangleup ABC$的外
接圆的劣弧$\hat{BC}$的中点,K是$\bigtriangleup BAC$的外角平分线与BC延长线的交点,在过点A且
垂直于BC的直线上取一点D(异于A),使得DM=AM.设$\bigtriangleup ADK$的外接圆与$\bigtriangleup ABC$外接圆相
交于点A及另一点T.证明:AT平分线段BC.
\begin{center}
    \includegraphics*[scale=0.35]{44}
\end{center}

5.如图,在等腰$\bigtriangleup ABC$中,$AB=BC$,I为内心,M为BI的中点,
P为边AC上一点,满足$AP=3PC$,PI延长线一点H满足$MH\perp PH$,Q为
$\bigtriangleup ABC$的外接圆上劣弧AB的中点.证明:$BH\perp QH$.
\begin{center}
    \includegraphics*[scale=0.30]{45}
\end{center}

6.如图,$\bigtriangleup ABC$为锐角三角形,AB<AC,M为BC边
的中点,点D和E分别为$\bigtriangleup ABC$的外接圆$\hat{BAC}$和$\hat{BC}$的中点,F为$\bigtriangleup ABC$的内
切圆在AB边上的切点,G为AE与BC的交点,N在线段EF上,满足$NB\perp AB$.
证明:若$BN=EM$,则$DF\perp FG$.
\begin{center}
    \includegraphics*[scale=1.0]{46}
\end{center}