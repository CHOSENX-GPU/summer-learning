\section{线段、角相等与图形相似、全等问题}
线段相等的证明方法:全等、等弧对等弦、比例线段法、代数法、张角定理、面积法等,
需要根据题目条件灵活运用。

角度相等的证明方法:全等或相似、四点共圆、角平分线与内心性质、三角法、鸡爪定理逆定理等。


1.在四边形ABCD中,对角线AC平分$\angle BAD$,在CD上取一点E,BE与AC
相交于F,延长DF交BC于G.证明:$\angle GAC=\angle EAC$.
~\\
~\\
~\\
~\\
~\\
~\\
~\\
~\\
~\\
~\\
~\\
~\\
~\\
~\\
~\\

2.设D为$\bigtriangleup ABC$的边AC上一点,E、F分别为线段BD和AC上的点,满足$\angle BAE=\angle CAF$.
再设P,Q为线段BC和BD上的点,使得$EP//QF//DC$.证明:$\angle BAP=\angle QAC$

\newpage

3.$\odot O_1,\odot O_2$相交于点M、N,设$l$为两圆的两条公切线中距离
点M较近的那条.$l$与两圆分别切于A、B两点.设经过点M且与$l$平行的直线与
两圆分别交于C、D(不同于M)两点.直线CA与DB交于点E,直线AN和CD相交于点
P,直线BN和CD相交于点Q.证明:EP=EQ.
\begin{flushleft}
    \includegraphics*[scale=0.35]{58}
\end{flushleft}
~\\

4.四边形ABCD内接于圆,P是AB的中点,$PE\perp AD,PF\perp BC,PG\perp CD$,
M是线段PG和EF的交点.证明:$ME=MF$.
\begin{flushleft}
    \includegraphics*[scale=0.35]{59}
\end{flushleft}

5.(牛顿线)完全四边形ABCD,三对角线的中点分别为M、N、T.证明:M、N、T三点共线.
\begin{flushleft}
    \includegraphics*[scale=0.35]{60}
\end{flushleft}

