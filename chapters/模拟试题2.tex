\section*{一试}
\subsection*{一.填空题(每题8分,共64分)}
1.若点$P(x,y)$在直线$x+3y-3=0$上移动,则函数$f(x,y)=3^x+9^y$的最小值为$\underline{5\sqrt[5]{\frac{27}{4}}}$.
~\\

2.由正$4n+2$边形的$4n+2$个顶点构成的梯形的个数为$\underline{2n(2n-1)(2n+1)}$.
~\\

3.设$O$是锐角$\bigtriangleup ABC$的外心,$AB=6,AC=10$.若$\overrightarrow{AO}=x\overrightarrow{AB}+y\overrightarrow{AC}$,且$2x+10y=5$,则$\cos{\angle BAC}=\underline{\frac{1}{3}}$.
~\\

4.数列$\{a_n \}$满足$a_1=\frac{1}{4},a_2=\frac{1}{5}$,且$a_1a_2+a_2a_3+\cdots+a_na_{n+1}=na_1a_{n+1}$对任何正整数$n$成立,则
$\frac{1}{a_1}+\frac{1}{a_2}+\cdots+\frac{1}{a_{97}}$的值为\underline{5044}.
~\\

5.已知正方体$ABCD-A_1B_1C_1D_1$的棱长为1,$O$为底面$ABCD$的中心,M,N分别是棱$A_1D_1,CC_1$的中点,则四面体$OMNB_1$的体积为$\underline{\frac{7}{48}}$.
~\\

6.关于$x$的方程$x^4-2ax^2-x+a^2-a=0$恰有两个不相等的实根,则实数$a$的取值范围是$\underline{(-\frac{1}{4},\frac{3}{4})}$.
~\\

7.已知点$F_1,F_2$分别是中心在坐标原点,对称轴为坐标轴的双曲线$C$的左、右焦点,过点$F_2$的直线$l$交双曲线的右支于$A,B$两点,点$I_1,I_2$分别是
$\bigtriangleup AF_1F_2,\bigtriangleup BF_1F_2$的内心.若双曲线$C$的离心率为2,$|I_1I_2|=\frac{9}{2}$,直线$l$的倾斜角的正弦值为$\frac{8}{9}$,
则双曲线$C$的方程为$\underline{\frac{x^2}{4}-\frac{y^2}{12}=1}$.
~\\

8.用$a_n$表示区间$\left[ 0,1 \right)$内不含数字9的$n$位小数的个数,$S_n$表示这些小数的和.则
$$\lim_{n\to \infty}\frac{S_n}{a_n}=\underline{\frac{4}{9}}.$$


9.(16分)已知$f(x)=\frac{x^4+kx^2+1}{x^4+x^2+1}(k\in R,x\in R)$.

(1)求$f(x)$的最大值和最小值;

(2)求所有的实数$k$,使得对任意的三个实数$a,b,c$,存在一个三角形,其三边长为$f(a),f(b),f(c)$.
\newpage
10.(20分)数列$\{a_n\},n\geq 1$定义为$a_1=1,a_2=4,a_n=\sqrt{a_{n-1}a_{n+1}+1}$.

(1)求证:数列$\{a_n \}$为整数列;

(2)求证:$2a_na_{n+1}+1,n\geq 1$是完全平方数.
\vspace{100mm}

11.(20分)在平面直角坐标系中,点$A,B,C$在双曲线$xy=1$上,满足$\bigtriangleup ABC$为等腰直角三角形.求$\bigtriangleup ABC$面积的最小值.

\newpage
\section*{二试}
%1.(40分)在$\bigtriangleup ABC$中,$\angle BCA$的角平分线与$\bigtriangleup ABC$的外接圆交于点R,与边
%BC的垂直平分线交于点P,与边AC的垂直平分线交于点Q,设点K、L分别是边BC、AC的中点.证明:
%$\bigtriangleup RPK$和$\bigtriangleup RQL$的面积相等.
%\begin{flushleft}
%\includegraphics*[scale=0.35]{66}
%\end{flushleft}
%\vspace{20mm}
1.设$\bigtriangleup ABC$的内心、关于$\angle A$的旁心分别为点$I,I_a$,AE是$\angle ABC$的外接圆的直径,$AD\perp BC$于点$D$.证明:$\angle IEI_a=\angle IDI_a$.
\begin{flushleft}
    \includegraphics*[scale=0.50]{67}
\end{flushleft}
\vspace{20mm}

2.(40分)已知$x_1,x_2,\cdots,x_n$是正实数,求证:
\[
    \frac{1}{1+x_1}+\frac{1}{1+x_1+x_2}+\cdots+\frac{1}{1+x_1+x_2+\cdots+x_n}<\sqrt{\frac{1}{x_1}+\frac{1}{x_2}+\cdots+\frac{1}{x_n}}
    \]


