\section{平面几何中的常见方法——代数法、三角法}

\subsection{代数法常用定理与结论}

比例线段、切线长公式(内心,旁心)、角平分线定理、梅涅劳斯定理、塞瓦定理、托勒密定理、
圆幂定理、调和性质等等。(还有一些几何结构中表现出的与线段长、线段比例相关的性质)

\subsection{代数法例题}
1.已知O、I分别为$\bigtriangleup ABC$的外心和内心,$BC=a,CA=b,AB=c$.问当且仅当$a,b,c$满足什么条件时,有$OI\perp BI$.
\begin{flushleft}
    \includegraphics*[scale=0.5]{51}
\end{flushleft}
~\\
~\\
~\\


2.AD是$\bigtriangleup ABC\angle BAC$的外角平分线,并交$\bigtriangleup ABC$的外界圆
于D,以CD为直径的圆分别交BC,AC于P,Q.证明:线段PQ把$\bigtriangleup ABC$的周长二等分.
\begin{flushleft}
    \includegraphics*[scale=0.5]{47}
\end{flushleft}
~\\
~\\
\newpage

3.$\bigtriangleup ABC$中,BE、CF为AC、AB边上的高,$FP\perp BC$于P,$FQ\perp AC$于Q,
$EM\perp BC$于M,$EN\perp AB$于N,且$FP+FQ=CF,\quad EM+EN=BE$.证明:$\bigtriangleup ABC$
为正三角形.
\begin{flushleft}
    \includegraphics*[scale=0.5]{48}
\end{flushleft}
~\\
~\\



4.设AB是$\odot O$的直径,过点A、B的切线分别为$l_a,l_b$,C是
圆周上任意一点,BC交$l_a$于K,$\angle CAK$的平分线交CK于H.设M是弧CAB的中点,
HM与$\odot O$交于点S,过点M的切线与$l_b$交于点T.证明:S、T、K三点共线.
\begin{flushleft}
    \includegraphics*[scale=0.5]{49}
\end{flushleft}
~\\
~\\
~\\

\subsection{三角法}
1.常用方法与公式:正弦定理、余弦定理、积化和差、和差化积、三倍角公式、三弦定理、张角定理等.

2.三角形中的恒等式:
\begin{equation*}
    \cos^2 A+\cos^2 B+\cos^2 C=1-2\cos A \cos B \cos C 
\end{equation*}

\begin{equation*}
    \tan A+\tan B+\tan C=\tan A \tan B \tan C
\end{equation*}

\begin{equation*}
    \tan{\frac{A}{2}} \tan{\frac{B}{2}}+\tan{\frac{B}{2}} \tan{\frac{C}{2}}+\tan{\frac{A}{2}} \tan{\frac{C}{2}}=1
\end{equation*}

3.设$r,R$分别为$\bigtriangleup ABC$的内切圆、外接圆半径,则有
\begin{equation*}
    \frac{r}{R}=4\sin{\frac{A}{2}} \sin{\frac{B}{2}} \sin{\frac{C}{2}}=\cos A + \cos B +\cos C-1
\end{equation*}

4.张角定理:若A、B、C分别是线束$Px,Py,Pz$上三点,则A、B、C三点共线的充要条件是:
$$\frac{\sin{\beta}}{PA} + \frac{\sin{\alpha}}{PB}=\frac{\sin{(\alpha+\beta)}}{PC}$$
\begin{center}
    \includegraphics*[scale=0.6]{50}
\end{center}

5.证明两角$\alpha,\beta$相等的方法:
\[
    \frac{\sin{(\alpha+\theta)}}{\sin{\alpha}}=\frac{\sin{(\beta+\theta)}}{\sin{\beta}}\quad \quad \quad (\theta \neq 0)
\]
\subsection{三角法例题}
1.求$\bigtriangleup ABC$旁切圆半径$r_a,r_b,r_c$与外接圆半径$R$的关系(用三角函数表示)
\newpage
2.O、I分别为O、I分别为$\bigtriangleup ABC$的外心和内心,AD是BC边上的高,O、I、D三点共线.证明:$\bigtriangleup ABC$外接圆半径$R$等于
BC边上旁切圆的半径$r_a$.
\begin{flushleft}
    \includegraphics*[scale=0.5]{52}
\end{flushleft}
~\\
~\\
~\\
~\\
~\\

3.AB为圆O的直径,直线$l$切圆O于A.
C、M、D在直线$l$上满足$CM=DM$,
又设BC、BD交圆O于P、Q,圆O切线PR、QR交于点R.
证明:B、M、R三点共线.
\begin{flushleft}
    \includegraphics*[scale=0.4]{53}
\end{flushleft}
\newpage

4.圆P与圆Q分别于圆O内切于点A、B,且圆P与圆Q外切.记圆P与圆Q的内公切线
与直线AB相交于点C.证明:OC平分线段PQ.
\begin{flushleft}
    \includegraphics*[scale=0.35]{54}
\end{flushleft}

5.已知$\bigtriangleup ABC$的一旁切圆与CA,CB延长线相切于点P、Q,另一旁切圆
与BA,BC延长线相切于点S,T.延长QP,TS交于点M.证明:$MA\perp BC$.
\begin{flushleft}
    \includegraphics*[scale=0.35]{55}
\end{flushleft}
~\\

6.已知$\odot I$的外切四边形ABCD,I在AC上的垂足为K.证明:$\angle BKC=\angle DKC$.
\begin{flushleft}
    \includegraphics*[scale=0.35]{57}
\end{flushleft}

