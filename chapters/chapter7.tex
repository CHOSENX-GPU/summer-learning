\section{直线平行或垂直问题}
\subsection{平行问题}
1.菱形ABCD的内切圆O与各边分别切于E、F、G、H,在弧EF与弧GH上
分别作圆O切线交AB于M,交BC于N,交CD于P,交DA于Q.
证明:$MQ//NP$.
\begin{flushleft}
    \includegraphics*[scale=0.35]{61}
\end{flushleft}


2.AD和CF是非锐角$\bigtriangleup ABC$的高,H和O分别是$\bigtriangleup ABC$
的垂心和外心,$M$是边AC的中点,直线BO交边AC于P,
直线BH和CF交于Q.证明:直线HM和PQ平行.
\begin{flushleft}
    \includegraphics*[scale=0.25]{62}
\end{flushleft}

3.在$\bigtriangleup ABC$中,AT为$\angle A$的平分线,D,E分别在AB、AC上,
且BD=CE.又BC、DE的中点分别为M和N.证明:$MN//AT$.
\begin{flushleft}
    \includegraphics*[scale=0.3]{63}
\end{flushleft}

\newpage
\subsection{垂直问题}
1.从等腰三角形$ABC$的底边$AC$的中点$M$作边$BC$的垂线$MH$,点$P$是$MH$的中点.证明:
$AH\perp BP$.
\begin{flushleft}
    \includegraphics*[scale=0.3]{68}    
\end{flushleft}

2.圆$O$的弦$AB$和$CD$交于$K$,过各弦的两端作圆的切线分别交于$P,Q$.求证:
$OK\perp PQ$.
\begin{flushleft}
    \includegraphics*[scale=0.35]{69}    
\end{flushleft}
~\\

3.$\bigtriangleup PAB$与$\bigtriangleup QAC$为$\bigtriangleup ABC$外的两个三角形,
满足$AP=AB,AQ=AC,\angle BAP=\angle CAQ$,线段BQ与CP相交于点R,设O是$\bigtriangleup BCR$的
外心.证明:$AO\perp PQ$.
\begin{flushleft}
    \includegraphics*[scale=0.35]{70}    
\end{flushleft}


