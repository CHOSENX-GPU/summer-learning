\section{圆中的比例线段、根轴}
\subsection{圆幂}
1.相交弦定理:圆内的两条相交弦被交点分成的两条线段的积相等。

2.切割线定理:从圆外一点引圆的切线和割线,切线长是这点到割线与圆交点的两条线段长
的比例中项

3.割线定理:从圆外一点引圆的两条割线,这一点到每条割线与圆的交点的两条线段长的积
相等

上述三个定理统称为圆幂定理:过一定点作两条直线与圆相交,则定点到每条直线与圆的交点的两条线段的
积相等,即它们的积为定值。
\begin{center}
    \includegraphics*[scale=0.4]{33}
\end{center}

4.点到圆的幂:点到圆心的距离为$d$,圆的半径为$r$,称$d^2-r^2$为定点到圆的幂。

当定点在圆内时,$d^2-r^2<0$

当定点在圆上时,$d^2-r^2=0$

当定点在圆外时,$d^2-r^2>0$

\subsection{根轴}
1.根轴:到两圆等幂的点的轨迹是与此二圆圆心连心线垂直的一条直线,这条直线称为两圆的根轴。

2.如果两圆相交,其根轴为两圆公共弦所在的直线。

3.如果两圆相切,其根轴就是过两圆切点的公切线。

4.三个圆,其两两的根轴或交于一点(根心),或互相平行。当三个圆两两相交时,三条公共弦所在的直线交于一点,
这一性质可以用于判断三线共点。
\begin{center}
    \includegraphics*[scale=0.3]{34}
\end{center}

\subsection{例题}
1.ABCD是圆$O$的内接四边形,延长$AB,DC$交于点$E$,
$EP,FQ$分别切圆$O$于$P,Q$.证明:$EP^2+FQ^2=EF^2$.

\begin{center}
    \includegraphics*[scale=0.3]{35}
\end{center}

2.自圆外一点P向圆O作切线PA、PB,A、B为切点,AB与PO
相交于C,弦EF过点C.证明:$\angle{APE}=\angle{BPF}$
\begin{center}
    \includegraphics*[scale=0.4]{36}
\end{center}

3.自圆外一点P向圆O引割线交圆于R、S两点,又作切线PA、PB,
A、B为切点,AB与PR相交于Q.证明:$\frac{1}{PR}+\frac{1}{PS}=\frac{2}{PQ}$
\begin{center}
    \includegraphics*[scale=0.35]{37}
\end{center}
\newpage
4.圆内接四边形ABCD的对角线交于点K,点M和N分别是
对角线AC和BD的中点,$\bigtriangleup ADM,\bigtriangleup BCM$的外接圆交于
点M、L,证明:K、L、M、N四点共圆.

~\\
~\\
~\\
~\\
~\\
~\\
~\\
~\\
~\\
~\\
~\\
~\\

5.$\bigtriangleup ABC$的内切圆与AB切于点C'
,设$\bigtriangleup ACC'$的内切圆分别与AB、AC切于点
$C_1,B_1$,$\bigtriangleup BCC'$的内切圆分别与AB、BC切于点$C_2,A_2$.
证明:$B_1C_1,A_2C_2,CC'$三线共点.
~\\
~\\
~\\
~\\
~\\
~\\
~\\
~\\
~\\
~\\
~\\

6.$\bigtriangleup ABC$中,E、F分别为AB、AC中点,CM、BN为高,
EF交MN于P,O、H分别为三角形的外心与垂心.证明:$AP\bot OH$.
\newpage

7.(欧拉定理)O、I分别为$\bigtriangleup ABC$的外心、内心,R、r分别为$\odot O,\odot I$的半径.
证明:$OI^2=R^2-2Rr$.
\begin{center}
    \includegraphics*[scale=0.5]{38}
\end{center}

8.已知D是$\bigtriangleup ABC$外接圆上任一点,作DE、DF与内切圆I都相切,交外接圆于E、F.
求证:EF也与内切圆I相切.
\begin{center}
    \includegraphics*[scale=0.5]{39}
\end{center}

9.$A_1A_2$是两个相离的圆$\odot O_1,\odot O_2$的外公切线,设$A_1A_2$的中点为$K$,过$K$作
$\odot O_1 ,\odot O_2$的切线$KB_1,KB_2$,$B_1,B_2$为切点.
直线$A_1B_1,A_2B_2$交于点$L$,直线$O_1O_2,KL$交于点$P$.
证明:$B_1,B_2,P,L$四点共圆.
\begin{center}
    \includegraphics*[scale=0.5]{40}
\end{center}