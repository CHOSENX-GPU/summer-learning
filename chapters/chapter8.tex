\section{点共线或线共点问题}
\subsection{点共线问题}

1.已知$O$是锐角三角形$ABC$的外心,$BE,CF$为$AC,AB$边上的高,自垂足$E,F$分别作
$AB,AC$的垂线,垂足为$G,H$.$EG,FH$相交于$K$.证明:$A,K,O$三点共线.
\begin{flushleft}
    \includegraphics*[scale=0.35]{72}
\end{flushleft}

2.四边形$BCEF$内接于圆$O$,其边$CE$与$BF$的延长线交于点$A$,由点$A$作圆$O$的两条切线
$AP,AQ$,切点分别为$P,Q$,$BE$与$CF$的交点为$H$.证明:$P,H,Q$三点共线.
\begin{flushleft}
    \includegraphics*[scale=0.35]{71}
\end{flushleft}

3.在锐角$\bigtriangleup ABC$的边$AB,AC$上分别取点$M,N$,分别以$BN,CM$为直径各作一圆,
两圆交于点$P,Q$.证明:$P,Q$以及$\bigtriangleup ABC$的垂心$H$共线.
\begin{flushleft}
    \includegraphics*[scale=0.35]{73}
\end{flushleft}

\newpage

\subsection{线共点问题}
4.设$P$是$\bigtriangleup ABC$内一点,满足$\angle APB-\angle ACB=\angle APC-\angle ABC$.
又设$D,E$分别是$\bigtriangleup APB,\bigtriangleup APC$的内心.证明:$AP,BD,CE$三线共点.
\begin{flushleft}
    \includegraphics*[scale=0.35]{74}
\end{flushleft}
~\\

5.四边形$ABCD$内接于圆$O$,对角线$AC$交$BD$于$P$.设$\bigtriangleup ABP,\bigtriangleup BCP,\bigtriangleup CDP,\bigtriangleup DAP$的外接圆
圆心分别为$O_1,O_2,O_3,O_4$.证明:$OP,O_1O_3,O_2O_4$三线共点.
\begin{flushleft}
    \includegraphics*[scale=0.35]{75}
\end{flushleft}
~\\

6.过$\bigtriangleup ABC$的两顶点$B,C$的圆分别与$AB,AC$相交于$C',B'$.设
$H,H'$分别为$\bigtriangleup ABC,\bigtriangleup AB'C'$的垂心.证明:
$BB',CC',HH'$三线共点.
\begin{flushleft}
    \includegraphics*[scale=0.25]{76}
\end{flushleft}

$(\mathbf{2 0 2 2} \cdot \mathbf{I M O} \cdot \mathbf{D 2} \cdot \mathbf{P 4})$ 
设凸五边形 $A B C D E$ 满足 $B C=D E$. 若在 $A B C D E$ 内存在一点 $T$ 使得 $T B=T D,T C=T E$ 且 $\angle A B T=\angle T E A$. 
直 线 $A B$ 分别与直线 $C D$ 和 $C T$ 交于点 $P$ 和 $Q$ ,且 $P, B, A, Q$ 在同一直线上按此顺序 排列; 
直线 $A E$ 分别与直线 $C D$ 和 $D T$ 交于点 $R$ 和 $S$ ,且 $R, E, A, S$ 在同一直线上 按此顺序排列. 证明: $P, S, Q, R$ 四点共圆.
\begin{flushleft}
    \includegraphics*[scale=0.5]{77}
\end{flushleft}