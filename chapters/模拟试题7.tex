\section*{一试}
\subsection*{一.填空题(每题8分,共64分)}
1.设$\alpha,\beta$为实常数,则$f(x)=3\sin{(x+\alpha)}+4\cos{(x+\beta)}$的最大值为\underline{\hbox to 20mm{}}.
~\\

2.计算:
\[
	\left[\dfrac{1}{4}\right]+\left[\dfrac{3}{4}\right]+\left[\dfrac{3^2}{4}\right]+\left[\dfrac{3^3}{4}\right]+\cdots+\left[\dfrac{3^{2022}}{4}\right]=\underline{\hbox to 30mm{}}.
\]

3.在$\bigtriangleup ABC$中,$\sin^2{A}+\sin^2{C}=2022\sin^2{B}$.则$\dfrac{(\tan{A}+\tan{C})\tan^2{B}}{\tan{A}+\tan{B}+\tan{C}}=$\underline{\hbox to 20mm{}}.
~\\

4.若关于$x$的方程$x^3-ax^2+(a^2-1)x-a^2+a=0$有三个不同的实根且按某种顺序可构成等差数列,则
实数$a$的值为\underline{\hbox to 20mm{}}.
~\\

5.把5封信随意装入写好地址与收信人的5个信封.则全部装错的概率为\underline{\hbox to 20mm{}}.
~\\

6.$\odot O:x^2+y^2=4$与$x$轴交于点$A,B$,过$\odot O$上的动点$P$作$\odot O$的切线,与过点
$B,A$的切线分别交于点$C,D$.则梯形$ABCD$对角线的交点$T$的轨迹方程为\underline{\hbox to 30mm{}}.
~\\

7.有一个半径为20的实心金属球,以某条直径为中心轴挖去一个半径为12的圆形的洞,再将余下的部分熔铸成一个新的实心球.那么新球的半径是\underline{\hbox to 20mm{}}.
~\\

8.已知正实数$a,b,c$满足:
$$
\begin{cases}
	a^2+ab+b^2=169\\
	b^2+bc+c^2=196\\
	c^2+ca+a^2=225\\
\end{cases}
$$

则$a+b+c=$\underline{\hbox to 30mm{}}.
~\\

\subsection*{二.解答题(本题满分56分)}
9.(16分)已知$M(1,0)$,直线$y=tx+1$与双曲线$\dfrac{x^2}{4}-\dfrac{y^2}{8}=1$有两个不同的交点$A,B$.设线段$AB$的中点为$P$,求直线
$PM$的斜率的取值范围.





\newpage
10.(20分)记数列${a_n}$的前$n$项和为$S_n$,且$\{ a_n \}$满足:
$a_1=\dfrac{1}{3},2a_na_{n-1}-3a_n+1=0(n\geq 2)$

(1)求数列${a_n}$的通项公式;
~\\

(2)证明:$\dfrac{3n-2}{6}<S_n<\dfrac{n}{2}.$

\vspace*{100mm}


11.(20分)实系数多项式$f(x)=a_4x^4+a_3x^3+a_2x^2+a_1x+a_0$,满足$-1\leq x \leq 1$时,$|f(x)|\leq 1$.
求$a_2$的最大值,并求$a_2$取得最大值时的所有实系数多项式$f(x)$.


