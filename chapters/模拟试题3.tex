\section*{一试}
\subsection*{一.填空题(每题8分,共64分)}

1.已知$\bigtriangleup ABC$的三个内角$\angle A,\angle B,\angle C$所对的边分别为$a,b,c$,且
满足$a\sin{A}\sin{B}+b\cos^2{A}=\sqrt{2}a$.则$\dfrac{b}{a}$的值是\underline{\hbox to 20mm{}}.
~\\

2.不等式$(x+1)(x+2)(x+3)>26x-2$的解集为\underline{\hbox to 20mm{}}.
~\\

3.已知函数$f(x)=|\log_2{x}|$,若实数$a,b(a<b)$满足$f(a)=f(b)$,则$a+2022b$的取值范围是\underline{\hbox to 20mm{}}.
~\\

4.已知等差数列$\{a_n\},\{b_n\}$的前$n$项之和分别为$S_n,T_n$,且$\dfrac{S_n}{T_n}=\dfrac{3n+2}{2n+1}$.则$\dfrac{a_7}{b_5}=\underline{\hbox to 20mm{}}.$
~\\

5.在平面直角坐标系中,以点$(1,0)$为圆心,$r$为半径作圆依次交抛物线$y^2=x$于$A,B,C,D$四点.若$AC$与$BD$的交点$F$恰为抛物线的焦点,
则$r=\underline{\hbox to 20mm{}}.$
~\\

6.四面体$ABCD$的对棱相等,$E,F$分别为$AB,CD$的中点,且$EF\perp AB,EF\perp CD,EF=6,AC=10,BC=6\sqrt{2}$.
则异面直线$AD,BC$的距离是\underline{\hbox to 20mm{}}.
~\\

7.若$a,b\in \mathbb{R}$,则$f(a,b)=\sqrt{2a^2-8a+10}+\sqrt{b^2-6b+10}+\sqrt{2a^2-2ab+b^2}$的最小值是\underline{\hbox to 20mm{}}.
~\\

8.设集合$A=\{1,2,3,\cdots,n\}$,用$S_n$表示$A$的所有非空真子集中各元素之和,$B_n$表示$A$的子集的个数,则
\[\lim_{n\to \infty }\dfrac{S_n}{n^2B_n}=\underline{\hbox to 20mm{}}\]

\subsection*{二.解答题(本题满分56分)}
9.(16分)已知奇函数$f(x)$在区间$(-\infty,0)$上是增函数,且$f(-2)=-1,f(1)=0$,当$x_1>0,x_2>0$时,有$f(x_1x_2)=f(x_1)+f(x_2)$,求
不等式$\log_2{|f(x)+1|}<0$的解集.

\newpage
10.(20分)已知数列$\{a_n\}$满足$a_1=1,a_{n+1}=\dfrac{a_n}{2}+2^{n-1}\cdot a_n^2(n\in \mathbf{N}*)$.
记$\displaystyle{S_n=\sum_{i=1}^n\dfrac{1}{2^ia_i+1}}$,求$\displaystyle{\lim_{n\to \infty}S_n}$的值.

\vspace{100mm}

11.(20分)已知抛物线$y^2=4x$的焦点为$F$,过点$F$作两条互相垂直的弦$AB,CD$,设$AB,CD$的中点分别为$M,N$.

(1)求证:直线$MN$必过定点,并求出定点坐标;

(2)分别以弦$AB,CD$为直径作圆,求两圆相交弦中点$H$的轨迹方程.