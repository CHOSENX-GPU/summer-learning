\section{圆锥曲线}
\subsection{椭圆}
\subsubsection{性质}
椭圆的第二定义:椭圆为平面内到一定点(焦点)$F$与一直线$l$的距离的比为$e$(离心率,$0<e<1$)的点的轨迹。

对椭圆$\dfrac{x^2}{a^2}+\dfrac{y^2}{b^2}=1$:

1.焦点:$F_1(-c,0),F_2(c,0)$,准线:$x=\pm \dfrac{a^2}{c}$,焦半径公式:$PF_1=a+ex_0,\quad PF_2=a-ex_0$。

2.椭圆上弦AB的中点为$M(x_0,y_0)$,则$k_{AB}k_{OM}=-\dfrac{b^2}{a^2}$.

3.AB为焦点弦,与$x$轴夹角为$\theta$,则有:
\[AF_2=\dfrac{b^2}{a+c\cos{\theta}},\quad BF_2=\dfrac{b^2}{a-c\cos{\theta}},\quad 
AB=\dfrac{2ab^2}{a^2-c^2\cos^2{\theta}}\]
\begin{center}
\includegraphics*[scale=0.8]{64}
\end{center}

4.椭圆的光学性质:椭圆上任一点M处的法线平分过该点的两条焦半径的夹角。

推广:自椭圆外任一点作椭圆的两条切线,则该点与一个焦点的连线平分该焦点
与切点弦的夹角。

5.蒙日圆:过圆$x^2+y^2=a^2+b^2$上任意点P作椭圆$\dfrac{x^2}{a^2}+\dfrac{y^2}{b^2}=1$
的两条切线,则两条切线垂直,反之也成立。

6.焦点三角形的面积:$\bigtriangleup PF_1F_2$为焦点三角形,$\angle F_1PF_2=2\theta$,
则$S_{\bigtriangleup PF_1F_2}=b^2\tan{\theta}$。

\subsubsection{例题}
1.把椭圆$\dfrac{x^2}{25}+\dfrac{y^2}{16}=1$的长轴AB均分为8段,过
每个分点作$x$轴的垂线交椭圆的上半部分于$P_1,P_2,P_3,\cdots,P_7$七个点,
F是椭圆的一个焦点,则$P_1F+P_2F+\cdots +P_7F$=\underline{\hbox to 30mm{}}.

\begin{center}
    \includegraphics*[scale=1.5]{65}
\end{center}

2.与1题椭圆相同,将$\angle AFB$均分为7份,即$\angle AFP_1=\angle P_1FP_2=\cdots=\angle P_6FB=\dfrac{\pi}{7}$,
则
\[\dfrac{1}{P_1F}+\dfrac{1}{P_2F}+\cdots+\dfrac{1}{P_6F}=\underline{\hbox to 30mm{}}.\]

3.已知椭圆$\dfrac{x^2}{25}+\dfrac{y^2}{9}=1$的右焦点是$F$,点$A(2,2)$在椭圆内,点$M$是椭圆
上的动点,那么$|MA|+|MF|$的最小值是\underline{\hbox to 30mm{}}.

~\\

4.已知椭圆$\dfrac{x^2}{a^2}+\dfrac{y^2}{b^2}=1(a>b>0)$与直线$x+y=1$交于$M,N$两点,且
$OM\perp ON$($O$为坐标原点),当椭圆的离心率$e\in \left[\dfrac{\sqrt{3}}{3},\dfrac{\sqrt{2}}{2}\right]$
时,椭圆长轴长的取值范围是\underline{\hbox to 30mm{}}.

~\\

5.在椭圆$\dfrac{x^2}{2}+y^2=1$中,弦长为2的弦的中点的轨迹方程为\underline{\hbox to 40mm{}}.
~\\

6.F为椭圆$\dfrac{x^2}{4}+y^2=1$的右焦点,过F作直线$l$交
椭圆于A、B两点,求$\bigtriangleup AOB$的面积最大值及此时$l$的倾斜角.

\vspace*{45mm}

7.当$a,b$满足什么条件时,椭圆$C_1$:$\dfrac{x^2}{a^2}+\dfrac{y^2}{b^2}=1$上任意一点P,
均存在以P为顶点、与圆$C_0$:$x^2+y^2=1$外切、与$C_1$内接的平行四边形?

\vspace{45mm}

8.在平面直角坐标系中,椭圆的方程为$\dfrac{x^2}{a^2}+\dfrac{y^2}{b^2}=1(a>b>0)$,$A_1,A_2$分别为椭圆的左右顶点,$F_1,F_2$分别为
椭圆的左右焦点,$P$为椭圆上不同于$A_1,A_2$的任意一点.若平面中的两个点$Q,R$满足$QA_1\perp PA_1,QA_2\perp PA_2,RF_1\perp PF_1,RF_2\perp PF_2$,
试确定线段$QR$的长度与$b$的大小关系,并给出证明.
\newpage

9.在平面直角坐标系中,$F_1,F_2$分别是椭圆$\dfrac{x^2}{2}+y^2=1$的左、右焦点.设
不经过焦点$F_1$的直线$l$与椭圆交于两个不同的点$A,B$,焦点$F_2$到直线$l$的距离为$d$.
如果直线$AF_1,l,BF_1$的斜率依次成等差数列,求$d$的取值范围.
\vspace{60mm}

10.过椭圆$\dfrac{x^2}{a^2}+\dfrac{y^2}{b^2}=1(a>b>0)$右焦点$F(1,0)$的直线(长轴除外)与椭圆
交于$M,N$两点,自$M,N$向右准线$l:x=4$作垂线,垂足分别为$M_1,N_1$.

(1)求此椭圆的方程.

(2)记$\bigtriangleup FMM_1,\bigtriangleup FM_1N_1,\bigtriangleup FNN_1$的面积
分别为$S_1,S_2,S_3$,是否存在$\lambda$,使得对任意的$a>0$,都有$S_2^2=\lambda S_1S_3$成立?若存在,
求出$\lambda$的值;若不存在,说明理由.
\vspace*{60mm}

11.已知椭圆$\dfrac{x^2}{24}+\dfrac{y^2}{16}=1$,直线$l:\dfrac{x}{12}+\dfrac{y}{8}=1$,$P$为
$l$上一点,射线$OP$交椭圆于$R$,又点$Q$在$OP$上且满足$|OQ|\cdot|OP|=|OR|^2$,
当点$P$在$l$上移动时,求点$Q$的轨迹.
\newpage

\subsection{双曲线}
\subsubsection{定义与基本性质}
1.第一定义:平面内与两定点$F_1,F_2$距离之差的绝对值是常数(小于$|F_1F_2|$)的点的轨迹为双曲线,
这两个定点叫作双曲线的焦点,两焦点的距离叫作焦距.
~\\

2.第二定义:平面内到一个定点$F$的距离与到一条定直线$l$的距离的比等于常数$e$($e>1$)的点的轨迹叫作双曲线,定点$F$为焦点,定直线$l$称为准线,常数$e$为离心率.
~\\

3.双曲线标准方程:$C_1:\dfrac{x^2}{a^2}-\dfrac{y^2}{b^2}=1(a>0,b>0,|x|\geq a)$或$C_2:\dfrac{y^2}{a^2}-\dfrac{x^2}{b^2}=1(a>0,b>0,|y|\geq a)$.
~\\

4.离心率:$e=\dfrac{c}{a}(e>1)$,准线:$C_1:x=\pm \dfrac{a^2}{c}\quad C_2:y=\pm \dfrac{a^2}{c}$,渐近线:$C_1:y=\pm \dfrac{b}{a}x \quad C_2:y=\pm \dfrac{a}{b}x$
~\\

5.焦半径公式:$C_1:\text{左支}:PF_1=-a-ex_1,PF_2=a-ex_1.\text{右支}:PF_1=a+ex_1,PF_2=-a+ex_1.$
$C_2:\text{下支}:PF_1=-a-ey_1,PF_2=a-ey_1.\text{上支}:PF_1=a+ey_1,PF_2=-a+ey_1.$\quad 通径长:$\dfrac{2b^2}{a}$
~\\

6.焦点三角形:$\bigtriangleup PF_1F_2$为焦点三角形,$\angle F_1PF_2=2\theta$,
则$PF_1\cdot PF_2=\dfrac{b^2}{\sin^2{\theta}},S_{\bigtriangleup PF_1F_2}=b^2\cot{\theta}$;$\bigtriangleup PF_1F_2$的内切圆在$x$轴上的切点为双曲线的顶点。
~\\

7.设$P,Q$是双曲线$\dfrac{x^2}{a^2}-\dfrac{y^2}{b^2}=1(b>a>0)$上的两点,$O$为中心,
若$OP\perp OQ$,则$\dfrac{1}{|OP|^2}+\dfrac{1}{|OQ|^2}=\dfrac{1}{a^2}-\dfrac{1}{b^2}$.
~\\

8.直线$Ax+By+C=0$与双曲线$\dfrac{x^2}{a^2}-\dfrac{y^2}{b^2}=1(a,b>0)$相交、相切、相离的充要条件是:
\[ A^2a^2-B^2b^2-C^2>=<0 \quad and \quad A^2a^2-B^2b^2\neq 0  \]

\subsubsection{例题}
1.过点$P(6,7)$,且与双曲线$\dfrac{x^2}{9}-\dfrac{y^2}{16}=1$相切的直线方程\underline{\hbox to 20mm{}}.
~\\

2.若直线$y=kx-1$与双曲线$\dfrac{x^2}{9}-\dfrac{y^2}{4}=1$仅有一个交点,则$k=\underline{\hbox to 20mm{}}.$
~\\

3.$F_1,F_2$为双曲线$\dfrac{x^2}{4}-\dfrac{y^2}{45}=1$的两个焦点,$P$是双曲线上一点,已知
$|PF_2|,|PF_1|,|F_1F_2|$成等差数列,且公差大于0,则$\angle F_1PF_2=$\underline{\hbox to 20mm{}}.
~\\

4.已知两圆$C_1:(x+4)^2+y^2=2,C_2:(x-4)^2+y^2=2$,动圆$M$与两圆$C_1,C_2$都相切,则
动圆圆心$M$的轨迹方程为\underline{\hbox to 20mm{}}.
~\\

5.点$M$到定点$A(0,-2\sqrt{2})$及定直线$y=\sqrt{2}$的距离之比是$\sqrt{2}:1$,则点$M$
的轨迹方程为\underline{\hbox to 20mm{}}.
~\\

6.过两曲线$x^2+2y^2-2=0,2x^2-y^2-2=0$的交点并且被$y$轴截得弦长为$\sqrt{13}$的圆锥曲线的方程为\underline{\hbox to 20mm{}}.
\newpage
7.设$P$为双曲线$\dfrac{x^2}{a^2}-\dfrac{y^2}{b^2}=1(a,b>0)$右支异于顶点的一点,$F_1F_2$分别为
其左右焦点.试求$\bigtriangleup PF_1F_2$的顶点$F_1$所对旁心的轨迹.
\vspace{40mm}

8.证明:过双曲线上任一点$P$作倾斜角为$\alpha$(定值)的直线$l$与双曲线两渐近线交于$Q,R$,则
$|PQ|\cdot |PR|$为定值.
\vspace{40mm}

9.设一圆与一等轴双曲线交于四点$A_1,A_2,A_3,A_4$,其中$A_1,A_2$是圆的直径的一对端点.证明:

(1)$A_3,A_4$是双曲线直径的端点.

(2)双曲线在$A_3$和$A_4$处的切线都垂直于$A_1A_2$.
\vspace{50mm}

10.设双曲线$xy=1$的两支$C_1,C_2$,正$\bigtriangleup PQR$的三顶点位于此双曲线上.

(1)求证:$P,Q,R$不能都在该双曲线的同一支上;

(2)设$P(-1,1)$在$C_2$上,$Q,R$在$C_1$上,求顶点$Q,R$的坐标.

一个有意思的结论:三顶点都在同一等轴双曲线上的三角形的垂心也在此等轴双曲线上.
\newpage

\subsection{抛物线}
\subsubsection{定义与性质}
1.定义:平面内到一定点$F$和一定直线$l$距离相等的点的轨迹叫作抛物线,点$F$为抛物线的焦点,直线$l$为抛物线的准线(定点F不在定直线上).由定义,抛物线的离心率$e=1$.
~\\

2.抛物线的标准方程:焦点在$x$轴正半轴上:$y^2=2px,p>0$,焦点坐标:$(\dfrac{p}{2},0)$,准线方程:$x=-\dfrac{p}{2}$,焦半径:$|PF|=x_0+\dfrac{p}{2}$.解题时,一般设点坐标为$P(\dfrac{y_0^2}{2p},y_0)$,方便计算;其余情况同理。
~\\

3.抛物线的光学性质:经过抛物线上的一点$M$,在抛物线内作一射线平行于抛物线的轴,此时经过这一点的法线平分这条射线和这一点焦半径的夹角。
~\\

4.抛物线的焦点弦:设过抛物线$y^2=2px(p>0)$的焦点$F$的直线与抛物线相交于$A(x_1,y_1),B(x_2,y_2)$,直线$OA,OB$的斜率分别为$k_1,k_2$,直线$l$的倾斜角
为$\alpha$.则有
$$y_1y_2=-p^2,x_1x_2=\dfrac{p^2}{4},k_1k_2=-4,|FA|=\dfrac{p}{1-\cos{\alpha}},|FB|=\dfrac{p}{1+\cos{\alpha}},|AB|=\dfrac{2p}{\sin^2{\alpha}}$$

$$\text{通径}=2p,S_{\bigtriangleup OAB}=\dfrac{p^2}{2\sin{\alpha}},\dfrac{1}{AF}+\dfrac{1}{BF}=\dfrac{2}{p}$$

5.直线$Ax+By+C=0$与抛物线$y^2=2px(p>0)$相交、相切、相离的充要条件是:$pB^2-2AC>=<0$.
~\\

6.$P(x_0,y_0)$是抛物线$y^2=2px(p>0)$上任一点,$F$为焦点,以$PF$为直径的圆必与$y$轴相切,切点是$Q(0,\dfrac{y_0}{2})$,并且直线
$PQ$是抛物线的切线.
~\\

7.$P,Q$是抛物线的焦点弦,以$PQ$为直径的圆必与抛物线的准线相切。设切点为$R$,抛物线焦点为$F$,则$RP,RQ$均与抛物线相切,$RP\perp RQ,RF\perp PQ$.
~\\

8.点$P$是抛物线上任意一点,过点$P$作准线$l$的垂线,垂足为$N$,抛物线过点$P$的切线交过顶点的切线于$Q$,交对称轴于$K$,$F$为其焦点,则四边形
$FPNK$是菱形;$F,Q,N$三点共线.
~\\

9.抛物线的交点为$F$,过抛物线上任意两点$P,Q$的切线交于$T$,则$\angle{TPF}=\angle{QTF},\angle{PTF}=\angle{TQF}$.
~\\

10.抛物线$y^2=2px(p>0)$,过点$P(2p,0)$的直线交抛物线于$A,B$两点,则$OA\perp OB$.
~\\
\subsubsection{例题}

1.平面上动点$P$到定点$A(2,0)$的距离比点$P$到$y$轴的距离大2,则动点$P$的轨迹方程应为\underline{\hbox to 20mm{}}.
~\\

2.以抛物线$(y+2)^2=4(x+1)$的焦点为顶点,且以直线$y=2$为准线的抛物线方程是\underline{\hbox to 20mm{}}.
~\\

\newpage

3.抛物线$y^2=2x$上到直线$x-y+3=0$的距离最短的点的坐标是\underline{\hbox to 20mm{}}.
~\\

4.若抛物线$y=x^2$上存在两点关于直线$y=m(x-3)$对称,则实数$m$的取值范围是\underline{\hbox to 20mm{}}.
~\\

5.过二次曲线$C_1:3x^2+8y^2-6x-32y=0$与$C_2:9x^2-16y^2-18x+24y=0$的交点的抛物线方程为\underline{\hbox to 20mm{}}.
~\\

6.过抛物线$y^2=2px(p>0)$上一点$P$作抛物线的切线交于$x$轴于点$Q$,点$F$为其焦点.求证:$|PF|=|QF|$.
\vspace{40mm}

7.求证:设抛物线$y^2=2px$上任意两点的弦的中点为$M$,弦两端点的切线的交点为$N$,则$MN$平行于$x$轴.
\vspace{40mm}

8.设抛物线$y=-x^2$与直线$y=3x-4$的两交点为$A,B$,点$P$在抛物线上由$A$到$B$运动.

(1)求当$\bigtriangleup ABC$面积最大时,$P$点的位置.

(2)证明:与$AB$平行的直线与抛物线相交于$C,D$两点时,直线$CD$被直线$x=x_0$等分.

\vspace{50mm}

9.已知抛物线$y^2=2px$及定点$A(a,b),B(-a,0),(ab\neq 0,b^2\neq 2qa)$.$M$是抛物线上的点,设
$AM,BM$与抛物线的另一交点分别是$M_1,M_2$.求证:当$M$点在抛物线上变动时(只要$M_1,M_2$存在且$M_1\neq M_2$),直线
$M_1M_2$横过定点,并求出定点的坐标.

\newpage

10.过抛物线$y^2=2px(p>0)$上的定点$A(a,b)$,引抛物线的两条弦$AP,AQ$.则
$AP\perp AQ$的充要条件是直线$PQ$过定点$M(2p+a,-b)$.
\vspace{50mm}

11.设$A$是抛物线$y^2=2px(p>0)$的对称轴上一点(位于抛物线内部),$B$是$A$关于$y$轴的对称点.求证:

(1)若过$A$的直线与这抛物线在对称轴两侧相交于$P,Q$两点,则$\angle PBA=\angle QBA$.

(2)若过$B$的直线与这抛物线在对称轴一侧相交于$P,Q$两点,则$\angle PAB+\angle QAB=180^{\circ}$.

\vspace{50mm}

\subsubsection{习题}
1.设曲线$C_1:\dfrac{x^2}{a^2}+y^2=1(a>0)$与$C_2:y^2=2(x+m)$在$x$轴上方
仅有一个公共点$P$.

(1)求实数$m$的取值范围(用$a$表示)

(2)$O$为原点,若$C_1$与$x$轴的负半轴交于点$A$,当$0<a<\dfrac{1}{2}$时,
试求$\bigtriangleup OAP$的最大值(用$a$表示).

\newpage

2.求函数$f(x)=\sqrt{x^4-3x^2-6x=13}-\sqrt{x^4-x^2+1}$的最大值.
\vspace{40mm}

3.求函数$f(x)=x^2-x+1+\sqrt{2x^4-18x^2+12x+68}$的最小值.
\vspace{40mm}

4.设$0<a<b$,过两定点$A(a,0)$和$B(b,0)$分别引直线$l$和直线$m$,使与抛物线
$y^2=4x$有四个不同的交点.当这四点共圆时,求此时直线$l$与直线$m$交点的轨迹.
\vspace{40mm}

5.抛物线$y^2=4ax(a>0)$的焦点为$A$,以$B(a+4,0)$为圆心,$|AB|$为半径在$x$轴
上方作半圆,设半圆与抛物线相交于不同的两点$M,N$,$P$为线段$MN$的中点.

(1)求$|AM|+|AN|$的值;

(2)是否存在这样的$a$,使$|AM|,|AP|,|AN|$成等差数列?
\vspace{40mm}

6.抛物线$y=ax^2-bx+b$与直线$y=a^2x(a>0)$相交于$P,Q$两点,$P,Q$的横坐标之差的绝对值为1.
设抛物线弧$PQ$上的点与直线的距离最大值为$d$.

(1)用$a$表示$d$.

(2)$a,b$取何值时,$d$值最大?
\newpage

\subsection{圆锥曲线综合}

1.直线系与圆系

(1)过直线$l_1:A_1x+B_1y+C_1=0$和直线$l_2:A_2x+B_2y+C_2=0$的交点的直线系:
$$A_1x+B_1y+C_1+\lambda (A_2x+B_2y+C_2)=0(\text{不包括}l_2)$$

(2)以直线$Ax+By+C=0$为根轴,与圆$x^2+y^2+Dx+Ey+F=0$等幂的圆系方程:
$$x^2+y^2+Dx+Ey+F+\lambda(Ax+By+C)=0$$.

(3)与圆$C_1:x^2+y^2+D_1x+E_1y+F_1=0$,圆$C_2:x^2+y^2+D_2x+E_2y+F_2=0$关于两圆根轴
等幂的圆系方程:
$$x^2+y^2+D_1x+E_1y+F_1+\lambda(x^2+y^2+D_2x+E_2y+F_2)=0(\text{不包括}C_2,\lambda=-1 \text{时为两圆的根轴})$$

(4)与直线$Ax+By+C=0$相切于点$M(x_0,y_0)$的圆系方程是:
$$ (x-x_0)^2+(y-y_0)^2+\lambda(Ax+By+C)=0 $$

(5)与圆$x^2+y^2+Dx+Ey+F=0$相切于点$M(x_0,y_0)$的圆系方程是:
$$ (x-x_0)^2+(y-y_0)^2+\lambda(x^2+y^2+Dx+Ey+F)=0 $$

2.二次曲线系

(1)二次曲线的一般表示:$f(x,y)=Ax^2+2Bxy+Cy^2+Dx+Ey+F=0(AB\neq 0)$
~\\

(2)退化的二次曲线:$(A_1x+B_1y+C_1)(A_2x+B_2y+C_2)=0$(表示两条直线)
~\\

(3)过四点的二次曲线系:经过不共线四点的两条二次曲线$f_1(x,y),f_2(x,y)$,则
$\lambda_1 f_1(x,y)+\lambda_2 f_2(x,y)=0$表示过这四点的所有二次曲线.
~\\

小结论:设$g_i=A_ix+B_iy+C_i$

(1)若三角形三边的方程为$g_i=0(i=1,2,3)$,则经过三角形三个顶点的二次曲线系为
$$g_1g_2+\lambda_1g_2g_3+\lambda_2g_1g_3=0$$

(2)若四边形(包括蝴蝶形等广义四边形)四条边的方程为$g_i=0(i=1,2,3,4)$,则经过四边形四个顶点的二次曲线系为
$$ g_1g_3+\lambda g_2g_4=0 $$

3.配极理论:对二次曲线$\varGamma:f(x^2,xy,y^2,x,y)=Ax^2+2Bxy+Cy^2+Dx+Ey+F=0 $,以及平面内ren
任意一点$P(x_0,y_0)$.点$P$关于曲线$\varGamma$的极线$l$的方程为:
$$ f(x_0x,y_0y,\dfrac{x_0y+y_0x}{2},\dfrac{x+x_0}{2},\dfrac{y+y_0}{2}) =0$$

当$P$为曲线上一点时,极线$l$即为过点$P$的切线;当$P$为曲线外一点时,极线$l$为$P$点关于
曲线的切点弦所在的直线.

共轭关系:若点$Q$在点$P$关于曲线$\varGamma$的极线上,则称$P,Q$两点共轭.此时,点$P$也在
$Q$关于曲线$\varGamma$的极线上.

在一般圆锥曲线中,我们在平面几何的圆中讨论的调和、交比、相切、完全四边形等有关性质,
一样成立,帕斯卡定理、德萨格定理、蝴蝶定理、$Candy$定理等与交比、射影性质相关的定理同样成立。
如圆锥曲线中的完全四边形构形,同样满足完全四边形对角线的交点关于该曲线共轭,是圆锥曲线题中定值定点问题常见
的解题思路。

配极变换、齐次坐标、仿射变换、广义反演变换等知识可以自己了解,在高联中并不需要,但可以作为打开思路的一种方式。

4.圆锥曲线的参数方程

(1)椭圆的参数方程:
$$
\begin{cases}
    x=a\cos{\alpha}\\
    y=b\sin{\alpha}
\end{cases}
$$

(2)双曲线的参数方程:
$$
\begin{cases}
    x=a\sec{\alpha}\\
    y=b\tan{\alpha}
\end{cases}
$$

(3)抛物线的参数方程:
$$
\begin{cases}
    x=2pt^2\\
    y=2pt
\end{cases}
$$

5.圆锥曲线的统一极坐标方程:
$$\rho=\dfrac{ep}{1-e\cos{\theta}}$$

$e$表示离心率,$p$表示焦点到准线的距离.